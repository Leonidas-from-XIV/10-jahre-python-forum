\documentclass{beamer}
\usepackage[T1]{fontenc}
\usepackage[utf8]{inputenc}
\usepackage{texments}
\usepackage{lmodern}
\usepackage{wasysym}
\usepackage{csquotes}
\usepackage{ccicons}

\mode<presentation>{\usetheme{Copenhagen}}
\usecolortheme{python}
\title{10~Jahre Python-Forum}
\author{Marek~Kubica}
\institute{PyCon DE 2012}

\begin{document}

\frame{
  \titlepage
  \vfill
  \begin{center}
    \ccby\\[2.5ex]
    {\tiny Dieses Werk bzw. Inhalt steht unter einer Creative Commons Namensnennung 3.0 Unported Lizenz.}
    \vspace*{-2.5ex}
  \end{center}
}

\begin{frame}{Was ist dieses Forum?}
  \begin{block}{Python-Forum}
    \begin{itemize}
      \item \url{http://www.python-forum.org}
      \item …
    \end{itemize}
  \end{block}
  \begin{block}{Warum Forum?}
    \begin{itemize}
      \item Einfacher erreichbar als Mailingliste, Newsgroup etc. da im Browser
      \item Strukturierter -- durchsuchbar, Unterkategorien
      \item Syntax-Highlighting \smiley
    \end{itemize}
  \end{block}
\end{frame}

\begin{frame}{Statistiken}
  TODO: Wie viele Nutzer, wie viele aktiv und wie lange, Beitragswachstum
\end{frame}

\begin{frame}{Warum sind die Leute zufrieden?}
  \begin{block}{Regeln}
    \begin{itemize}
      \item Keine visuell ablenkenden Gimmicks
      \item Wenig Moderationseingriffe
      \item Keine Statuskämpfe: neue Moderatoren aus der Community
    \end{itemize}
  \end{block}
\end{frame}

\begin{frame}{Was man nicht machen sollte}
  TODO: aus dem Leben eines Trolls
  \enquote{Pyhton}
\end{frame}

\begin{frame}{Erfolgreiche Projekte aus dem Forum}
  \begin{itemize}
    \item Bottle Micro-Framework, \url{http://bottlepy.org}
    \item Deutschsprachige Übersetzung des Python 3.x-Tutorials \url{http://tutorial.pocoo.org}
    \item simplemail.py, Helfer für einfachen Versand von E-Mails
  \end{itemize}
\end{frame}

\end{document}
