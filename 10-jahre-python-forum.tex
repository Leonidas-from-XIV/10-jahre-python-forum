\documentclass{beamer}
\usepackage[ngerman]{babel}
\usepackage{fontspec}
\usepackage{xltxtra}
\usepackage{texments}
\usepackage{unicode-math}
\usepackage{csquotes}
\usepackage{ccicons}
\usepackage{tikz}

\mode<presentation>{\usetheme{Copenhagen}}
\usecolortheme{python}
\title{10~Jahre Python-Forum}
\author{Marek~Kubica}
\institute{PyCon DE 2012}
\date{30.~Oktober~2012}

\begin{document}

\setmainfont{Latin Modern Roman}
\setsansfont{Latin Modern Sans}
\setmonofont{Latin Modern Mono}
\setmathfont{Latin Modern Math}

\frame{
  \titlepage
  \vfill
  \begin{center}
    \ccby\\[2.5ex]
    {\tiny Dieses Werk bzw. Inhalt steht unter einer Creative Commons Namensnennung 3.0 Unported Lizenz.}
    \vspace*{-2.5ex}
  \end{center}
}

\newcommand{\dejavu}[1]{{\setsansfont{DejaVu Sans}\selectfont #1}}

\begin{frame}[plain]
  \begin{tikzpicture}[remember picture,overlay]
    \node[at=(current page.center)] {
      \includegraphics[width=\paperwidth]{screenshot}
    };
  \end{tikzpicture}
\end{frame}

\begin{frame}{Was ist dieses Forum?}
  \begin{block}{Python-Forum}
    \begin{itemize}
      \item \url{http://www.python-forum.de}
      \item …
    \end{itemize}
  \end{block}
  \begin{block}{Warum Forum?}
    \begin{itemize}
      \item Einfacher erreichbar als Mailingliste, Newsgroup etc. da im Browser
      \item Strukturierter – durchsuchbar, Unterforen
      \item Syntax-Highlighting \dejavu{☺}
    \end{itemize}
  \end{block}
\end{frame}

\begin{frame}{Statistiken}
  TODO: Wie viele Nutzer, wie viele aktiv und wie lange, Beitragswachstum
\end{frame}

\begin{frame}{Warum sind die Leute zufrieden?}
  \begin{block}{Regeln}
    \begin{itemize}
      \item Keine visuell ablenkenden Gimmicks
      \item Wenig Moderationseingriffe
      \item Keine Statuskämpfe: neue Moderatoren aus der Community
      \item Keine Threads schließen
    \end{itemize}
  \end{block}
  \begin{block}{Hilfestellung}
    \begin{itemize}
      \item Auf die meisten Fragen gibt es (hilfreiche) Antworten
      \item Anfängern wird viel Hilfestellung gegeben – sofern sie lernwillig sind
      \item \enquote{Dumme} Ideen werden hinterfragt
      \item Perfektionismus bei Antworten
    \end{itemize}
  \end{block}
\end{frame}

\begin{frame}{Ich kann Python schon, was bringt mir das Forum?}
  \begin{block}{Sozial}
    \begin{itemize}
      \item Kontakt zu anderen Pythoneers – $\mu$Py entstand aus einem Forumtreffen
      \item Wohliges Gefühl etwas an die Community zurückgegeben zu haben \dejavu{😎}
    \end{itemize}
  \end{block}
  \begin{block}{Technisch}
    \begin{itemize}
      \item Code-Reviews
      \item Projektvorstellungen
      \item Architekturdiskussionen
    \end{itemize}
  \end{block}
\end{frame}

\begin{frame}{Wie starte ich einen Thread und bekomme sinnvoll Hilfe?}
\end{frame}

\begin{frame}{Was man nicht machen sollte}
  \begin{itemize}
    \item \enquote{Pyhton} \dejavu{😨}
    \item \enquote{Hier ist die Aufgabe aus dem Unterricht und ich komm nicht weiter,
      macht mal!}
    \item \enquote{Ich habe hier dieses Skript für Cinemafield4D und möchte es
      anpassen, will aber kein Python lernen. Hilfe!!!}
    \item Betreff: \enquote{Einfache Frage}
    \item \enquote{Meine Python hat eine Maus gefressen und nun ist der Bauch extrem geschwollen} \dejavu{😰}
    \item Tipps ignorieren
  \end{itemize}
\end{frame}

\begin{frame}{Was man nicht machen sollte}
  TODO: aus dem Leben eines Trolls
\end{frame}

\begin{frame}{Erfolgreiche Projekte aus dem Forum}
  \begin{itemize}
    \item Bottle Micro-Framework, \url{http://bottlepy.org}
    \item Deutschsprachige Übersetzung des Python 3.x-Tutorials \url{http://tutorial.pocoo.org}
    \item \texttt{simplemail.py}, Helfer für einfachen Versand von E-Mails
  \end{itemize}
\end{frame}

\begin{frame}{Foren-Humor}
\end{frame}

\begin{frame}{Zum Schluss}
  \begin{columns}[t]
    \column{0.5\textwidth}
  \begin{block}{Danksagungen an}
    \begin{itemize}
      \item Carsten Sandtner, der das Forum gegründet hat
      \item Admins und Moderatoren
      \item die Regulars
      \item unsere User!
    \end{itemize}
  \end{block}
    \column{0.5\textwidth}
  \begin{block}{Treffen}
    \begin{itemize}
      \item Erstes Forentreffen, zum 10. Jubiläum
      \item Heute (30.~Oktober~2012), 18:00 - 20:00
      \item Foyer des KUBUS
      \item Getränke und Snacks vorhanden
      \item Offen für alle, egal ob Mitglieder oder nicht
      \item Erscheint zahlreich \dejavu{☺}
    \end{itemize}
  \end{block}
  \end{columns}
\end{frame}

\end{document}
