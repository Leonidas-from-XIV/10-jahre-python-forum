\documentclass{beamer}
\usepackage[ngerman]{babel}
\usepackage{fontspec}
\usepackage{xltxtra}
\usepackage{texments}
\usepackage{unicode-math}
\usepackage{csquotes}
\usepackage{ccicons}
\usepackage{tikz}

\mode<presentation>{\usetheme{Copenhagen}}
\usecolortheme{python}
\title{10~Jahre Python-Forum}
\author{Marek~Kubica}
\institute{PyCon DE 2012}
\date{30.~Oktober~2012}

\begin{document}

\setmainfont{Latin Modern Roman}
\setsansfont{Latin Modern Sans}
\setmonofont{Latin Modern Mono}
\setmathfont{Latin Modern Math}

\frame{
  \titlepage
  \vfill
  \begin{center}
    \ccby\\[2.5ex]
    {\tiny Dieses Werk bzw. Inhalt steht unter einer Creative Commons Namensnennung 3.0 Unported Lizenz.}
    \vspace*{-2.5ex}
  \end{center}
}

\newcommand{\dejavu}[1]{{\setsansfont{DejaVu Sans}\selectfont #1}}

\begin{frame}{Warum dieser Talk?}
  \begin{block}{PyCon DE 2011}
    Vortrag \enquote{Python Community im deutschsprachigen Raum} erwähnt das
    Python-Forum
  \end{block}
  \begin{block}{2012}
    \begin{itemize}
      \item 10-jähriges Jubiläum des Python-Forums
      \item Erstes Usertreffen zum Geburtstag
      \item PyCon DE 2012
    \end{itemize}
  \end{block}
\end{frame}

\begin{frame}[plain]
  \begin{tikzpicture}[remember picture,overlay]
    \node[at=(current page.center)] {
      \includegraphics[width=\paperwidth]{screenshot}
    };
  \end{tikzpicture}
\end{frame}

\begin{frame}{Was ist dieses Forum?}
  \begin{block}{Python-Forum}
    \begin{itemize}
      \item \url{http://www.python-forum.de}
      \item Unterforen zu Installation, Webframeworks, Netzwerken, Datenbanken,
	Interoperabilität
      \item GUI-Toolkits, Ideen, Links und Tutorials, Verbesserungsvorschläge
      \item Offtopic, …
    \end{itemize}
  \end{block}
  \begin{block}{Warum Forum?}
    \begin{itemize}
      \item Einfacher erreichbar als Mailingliste, Newsgroup etc. da im Browser
      \item Strukturierter – durchsuchbar, mit Unterforen, in Suchmaschinen auffindbar
      \item Vor allem für \emph{Neueinsteiger} wichtig
      \item Syntax-Highlighting \dejavu{☺}
    \end{itemize}
  \end{block}
\end{frame}

\begin{frame}{Statistiken}
  \begin{block}{Wachstum}
    \begin{enumerate}
      \item 30.~Juli~2002: Gründung, 1 User (piddon)
      \item Mai~2005: 600 User, 14.000 Beiträge
      \item April~2006: 1.500 User, 35.000 Beiträge
      \item März~2007: 2.800 User, 60.000 Beiträge
      \item August~2007: 3.400 User, 72.000 Beiträge
      \item September~2008: 5.000 User, 107.000 Beiträge
      \item Juni~2009: 6.000 User, 134.000 Beiträge
      \item November~2012: 9.200 User, 225.000 Beiträge
    \end{enumerate}
  \end{block}
  \begin{block}{User}
    Normale Nutzer, Gruppe von Regulars, 6~Moderatoren aus dieser Gruppe,
    inzwischen 31~Personen mit über 1000~Posts, zwei davon mit über
    10.000~Posts.
  \end{block}
\end{frame}

\begin{frame}{Statistiken}
  \begin{block}{Zum Vergleich}
    \begin{itemize}
      \item de.comp.lang.python: 286 User, 1762 Themen\footnote{Google Groups, total
        wissenschaftlich, wissenschon!!1!}
      \item python-de: 8052 Beiträge\footnote{Gmane}
    \end{itemize}
  \end{block}
  \enquote{Lies, damned lies, and statistics} — Mark Twain
\end{frame}

\begin{frame}{Warum sind die Leute zufrieden?}
  \begin{block}{Hilfestellung}
    \begin{itemize}
      \item Angenehmer Umgangston
      \item Auf die meisten Fragen gibt es (hilfreiche) Antworten
      \item Anfängern wird viel Hilfestellung gegeben – sofern sie lernwillig sind
      \item \enquote{Dumme} Ideen werden hinterfragt
      \item Perfektionismus bei Antworten
    \end{itemize}
  \end{block}
  \begin{block}{Regeln}
    \begin{itemize}
      \item Keine visuell ablenkenden Gimmicks wie animierte Signaturen
      \item Wenig Moderationseingriffe
      \item Keine Statuskämpfe: neue Moderatoren aus der Community
      \item Kein Schließen von Threads
    \end{itemize}
  \end{block}
\end{frame}

\begin{frame}{Ich kann Python schon, was bringt mir das Forum?}
  \begin{block}{Technisch}
    \begin{itemize}
      \item Code-Reviews
      \item Projektvorstellungen
      \item Architekturdiskussionen
    \end{itemize}
  \end{block}
  \begin{block}{Sozial}
    \begin{itemize}
      \item Kontakt zu anderen Pythoneers – $\mu$Py entstand aus einem Forumtreffen
      \item Wohliges Gefühl etwas an die Community zurückgegeben zu haben \dejavu{😎}
      \item Netter Stammtisch
    \end{itemize}
  \end{block}
\end{frame}

\begin{frame}{Wie starte ich einen Thread und bekomme sinnvoll Hilfe?}
  Ziel: Potentielle Antwortgeber motivieren an der Diskussion teilzunehmen.
  \begin{enumerate}
    \item Suchfunktion nutzen
    \item Passendes Unterforum aussuchen
    \item Gute Überschrift wählen
    \item Problem in halbwegs verständlichem Deutsch erklären
    \item Eigenen Ansatz erklären
    \item Code posten, mit Syntax Highlighting
    \item Auf Nachfragen reagieren
    \item Freundlich sein
  \end{enumerate}
  Ein \enquote{Danke!} danach wirkt Wunder!
\end{frame}

\begin{frame}{Was man nicht machen sollte}
  \begin{itemize}
    \item \enquote{Pyhton} \dejavu{😨}
    \item \enquote{Hier ist die Aufgabe aus dem Unterricht und ich komm nicht weiter,
      macht mal!}
    \item \enquote{Ich habe hier dieses Skript für Cinemafield4D und möchte es
      anpassen, will aber kein Python lernen. Hilfe!!!}
    \item Betreff: \enquote{Einfache Frage}/\enquote{Problem}
    \item \enquote{Meine Python hat eine Maus gefressen und nun ist der Bauch extrem geschwollen} \dejavu{😰}
    \item Tipps ignorieren
    \item Tippgeber beschimpfen
  \end{itemize}
\end{frame}

\begin{frame}{Erfolgreiche Projekte aus dem Forum}
  \begin{itemize}
    \item Bottle Micro-Framework, \url{http://bottlepy.org}
    \item Deutschsprachige Übersetzung des Python 3.x-Tutorials \url{http://tutorial.pocoo.org}
    \item \texttt{simplemail.py}, Helfer für einfachen Versand von E-Mails
  \end{itemize}
\end{frame}

\begin{frame}{Foren-Humor}
  \begin{itemize}
    \item \enquote{python funny cats} — Wenn jemand nicht ordentlich sucht, gutes
      Google-Ranking für den Begriff.
    \item \enquote{Japanische Flagge} — Hausaufgabe ist es, eine Bilddatei mit
      Japanischer Flagge zu generieren. Code Golf, kürzeste Lösung 84 Bytes
    \item Hausaufgaben-betteln generell: als Antworten kommen Lösungen in obfuskiertem
      Python, C, Perl, Ruby, Haskell, CoffeeScript, Erlang, C64 BASIC, Elisp, Scheme,
      Clojure, OCaml, Io, x86 Assembler, Java, Erlang, Vala, FORTH, Scratch
  \end{itemize}
\end{frame}

\begin{frame}{Zum Schluss}
  \begin{columns}[t]
    \column{0.5\textwidth}
  \begin{block}{Danksagungen an}
    \begin{itemize}
      \item Carsten Sandtner, der das Forum gegründet hat
      \item Admins und Moderatoren
      \item die Regulars
      \item unsere User!
    \end{itemize}
  \end{block}
    \column{0.5\textwidth}
  \begin{block}{Treffen}
    \begin{itemize}
      \item Erstes Forentreffen, zum 10. Jubiläum
      \item Heute (30.~Oktober~2012) ab 17:30
      \item Foyer des KUBUS
      \item Getränke und Snacks vorhanden
      \item Offen für alle, egal ob Mitglieder oder nicht
      \item Erscheint zahlreich \dejavu{☺}
    \end{itemize}
  \end{block}
  \end{columns}
\end{frame}

\end{document}
